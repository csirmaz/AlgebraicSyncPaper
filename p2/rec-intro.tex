
If we start with two copies of a filesystem $\FS$,
and two different sequences are applied to the copies to yield $\FS_A=A\aFS$
and $\FS_B=B\aFS$, then our aim is to define sequences of commands $\reca$ and $\recb$
so that $\recb\aFS_A$ and $\reca\aFS_B$ would be as close to each other as possible.

We work based on the assumption that to achieve this, we need
to apply to $\FS_B$ the commands that have been applied to $\FS_A$, and \emph{vice versa}.
As some commands may have been applied to both filesystems, our first approximation
is to apply the commands in $\bmanp$ to $\FS_A$ and those in $\ambnp$ to $\FS_B$.
This, however, will break both filesystems if there have been incompatible updates
in $A$ and $B$. 
Our aim is therefore to provide an algorithm that selects the commands 
$\reca \subset A\setminus B$
and $\recb \subset B\setminus A$ 
so that $\reca\aFS_B\neq\fsbroken$ and $\recb\aFS_A\neq\fsbroken$,
and show that these are the longest sequences with this property.

We will work based on update sequences $A$ and $B$ that are simple.
If the reconciliation algorithm is to work on arbitrary sequences,
\cref{can_simplify} can be used to convert these sequences to simple ones.

\begin{mydef}[Reconciliation]\mlabel{def:reconciliation}
For two simple sequences $A$ and $B$,
$\reca$ is a valid ordering of
the largest subset of $A\setminus B$
that is independent of $B\setminus A$:
\[ \reca = \ords{\alpha \whr \alpha\in \ambnp  \mbox{~and~}  \alpha\indep \bmanp }. \]
$\recb$ can be obtained in a similar way by reversing $A$ and $B$
in the definition.
\end{mydef}

Note that $\alpha\in\reca$ is not an assertion command, and $\bmanp$ contains no assertion commands,
and $\alpha\not\in\bmanp$, therefore $\alpha\indep\bmanp$ is equivalent to
the node of $\alpha$ being incomparable to the nodes of the commands in $\bmanp$.
Therefore whether two commands are independent 
for the purposes of reconciliation
is easy to determine programmatically.
Then, to generate the sequence $\reca$ from the resulting set, 
apply topological sorting according to $\orderrel$ as defined in \cref{ordering}.


\myskip
We aim prove that $\reca$ and $\recb$ can be applied to the replicas,
that is, without loss of generality,
\[ \reca\aFS_B\neq\fsbroken. \]
So that we can formalize statements needed to prove this,
we introduce the partial order $\worksmeqsign$ that describes conditions under which
sequences of commands work, that is, do not break a filesystem.
