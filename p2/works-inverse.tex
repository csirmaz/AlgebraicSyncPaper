
% Works
% -----

\subsection{Domains of Sets of Command Sequences}

\begin{mydef}[Sets of sequences, their domains]
As we will frequently refer to sets of fequences,
we will use calligraphic letters (e.g. $\sqs{A}$, $\sqs{B}$, $\sqs{C}$ and $\sqs{S}$)
to denote such sets for brevity.
We will write $\Dom{\sqs{A}}$ for $\bigcap_{A\in\sqs{A}} \Dom{A}$,
and $\sqs{A}\cc\sqs{B}$ for $\{A\cc B\whr A\in\sqs{A}, B\in\sqs{B}\}$.
\end{mydef}

% works operators
% ---------------

\begin{mydef}[$\worksmeqsign$]\mlabel{def_works}
For two sets of sequences $\sqs{A}$ and $\sqs{B}$
we write $\worksm{\sqs{A}}{\sqs{B}}$ iff $\Dom{\sqs{A}} \subseteq \Dom{\sqs{B}}$;
that is, iff all sequences in $\sqs{B}$ are defined on (do not break)
all filesystems on which sequences in $\sqs{A}$ are defined.
\end{mydef}

When $\sqs{A}$ or $\sqs{B}$ contains a single sequence,
we leave out the curly brackets and write,
e.g. $A\cc\sqs{B}$ to mean $\{A\}\cc\sqs{B}$,
or $\worksm{A}{B}$ to mean $\worksmbb{A}{B}$.
Also, we write $\worksmnil{\sqs{A}}$ to mean that
there are filesystems on which all sequences in $\sqs{A}$ are defined.
If $\sqs{A}$ contains a single sequence only, $A$, this is equivalent to $\wrks{A}$.

We can see that $A\eqext B$ implies $\worksm{A}{B}$, as the latter
only requires that $B$ is defined where $A$ is defined, 
while the former also requires
that where they are defined, their effect is the same.

The following claims follow from the definition:

\begin{myclm}\mlabel{worksextpostfix}
% An inference rule in the algebra
$\forall A,S: \worksm{A\cc S}{A}$, that is, if a sequence is defined,
its initial segments are also defined.
\end{myclm}

\begin{myclm}\mlabel{works_restricted}
$\forall \sqs{A},\sqs{B},S: \worksm{\sqs{A}}{\sqs{B}} \Rightarrow \worksm{S\cc \sqs{A}}{S\cc \sqs{B}}$,
that is,
the relationship between the domains of the sets $\sqs{A}$ and $\sqs{B}$ 
does not change if both are prefixed by a sequence $S$.
\end{myclm}
\begin{proof}
This is because the function $S$ is a binary relation between filesystems,
and we can treat its inverse relation as a one-to-many mapping between filesystems
that maps $\Dom{\sqs{A}}$ to $\Dom{S\cc\sqs{A}}$ and $\Dom{\sqs{B}}$ to $\Dom{S\cc\sqs{B}}$.
As such a one-to-many mapping maps a subset of a set to a subset of the image of the set,
if $\Dom{\sqs{A}}\subseteq\Dom{\sqs{B}}$, then $\Dom{S\cc\sqs{A}}\subseteq\Dom{S\cc\sqs{B}}$.
\end{proof}

We proceed by proving the following lemmas.

\begin{mylem}\mlabel{combine_independent_commands}
The combination of independent commands is defined on all filesystems
where both of the commands are defined:
\[ \alpha\indep \beta \Rightarrow \worksmbx{\alpha, \beta}{\alpha\cc \beta}. \]
\end{mylem}
\begin{proof}
% In the following proof we reach back to our filesystem model.
The proposition follows from \cref{independent_details},
and we follow the three cases listed there.

The first case is that $n\unrel m$ where $\alpha=\cxynv$, $\beta=\czwmv$.

TODO CONTINUE

Using an inverse proof we assume that for a filesystem $\FS$
both $\cxynv\aFS$ and $\czwmv\aFS$ are defined, but $(\cxynv\cc\czwmv)\FS$ is broken.
We know a command breaks a filesystem either if
its input type is not compatible with the filesystem,
or if the filesystem ceases to have the tree property.

We also know that whether a filesystem has the tree property
depends 

The second case is that the two commands are the same and are assertion or replacement
commands, when the proposition is trivially true.
Finally, the third case is that the two commands are on comparable nodes and one of them
is an assertion command, when the proposition is again trivially true.

TODO END

Let $\alpha=\cxynv$ and $\beta=\czwmv$.
The proof is by contradiction;
assume that there is a filesystem $\FS$ for which
$\cxynv\aFS\neq\fsbroken$ and $\czwmv\aFS\neq\fsbroken$, but
$(\cxynv\cc \czwmv)\aFS=\fsbroken$.
We know $\cxynv\aFS\neq\fsbroken$ so it must be applying 
$\czwmv$ that breaks it.

We note that as the two commands are independent, they commute, and so
the proposition is symmetric.
Therefore, if either of the commands is an assertion command,
without loss of generality we can assume $\cxynv$ is an assertion command.
If so, 
$[\cxynv\aFS(m)]=[\FS(m)]$ even if $n=m$.
This means $\czwmv$ cannot break $\cxynv\aFS(m)$, which is a contradiction.

In the remaining cases neither command is an assertion command.

Applying a command can only result in a broken filesystem in two cases.
First, if the input type does not match the filesystem,
but we know $[\FS(m)]=[w]$ and so
$[\cxynv\aFS(m)]=[w]$ as based on \cref{incomparable_is_independent}, $n\neq m$.
Second, if the new filesystem violates the tree property.
This again cannot be the case because we also know that $n\unrel m$
and the tree property only depends on the types of values at the parent and children of $m$,
which therefore cannot be changed by $\cxynv$.
\end{proof}

\cref{combine_independent_commands} extends to sequences as well:

\begin{mylem}\mlabel{combine_independent_sequences}
The combination of independent sequences is defined on all filesystems
where both of the sequences are defined:
\[ S\indep T \Longrightarrow \worksmbx{S,T}{S\cc T}. \]
\end{mylem}
\begin{proof}
Assume that there is a filesystem $\FS$ so that
$S\aFS\neq\fsbroken$ and $T\aFS\neq\fsbroken$, but
$(S\cc T)\FS=\fsbroken$.

From \cref{def_indep} we know that
the commands in $S$ and $T$ pairwise commute, and so any sequence
that contains the commands from $S$ and $T$ and preserve their original partial order
is equivalent to $S\cc T$ on all filesystems.

Let the command in $T$ that breaks $\FS$ when applying $S\cc T$ be $t$
so that $T=T_0\cc t\cc T_1$.
It is still true that $(T_0 \cc t)\FS\neq\fsbroken$,
and by definition $(S\cc T_0)\FS\neq\fsbroken$,
but $(S\cc T_0\cc t)\FS=\fsbroken$.
Also, from above we know that $S\cc T_0\equiv T_0\cc S$
and so $(T_0 \cc S)\FS\neq\fsbroken$.

If we denote the first command in $S$ with $s_1$,
this means that $(T_0 \cc s_1)\FS\neq\fsbroken$,
which we can combine with $(T_0 \cc t)\FS\neq\fsbroken$, $t\indep s_1$ and
\cref{combine_independent_commands}
(using $T_0\FS$ as the reference filesystem)
to arrive at $(T_0 \cc s_1\cc t)\FS\neq\fsbroken$.

We can repeat this step for $s_2$, the next command in $S$,
and from 
$(T_0 \cc s_1\cc t)\FS\neq\fsbroken$
and
$(T_0 \cc s_1\cc s_2)\FS\neq\fsbroken$
arrive at
$(T_0 \cc s_1\cc s_2\cc t)\FS\neq\fsbroken$.
This can be repeated until $S$ is exhausted and we get
$(T_0 \cc S\cc t)\FS\neq\fsbroken$, which is a contradiction.
\end{proof}

We also prove the following:

\begin{mylem}\mlabel{worksinputmatch}
If $S$ and $T$ are minimal sequences, $\worksmnilb{S,T}$,
and there are commands $\cxynv\in S$ and $\czwnv\in T$ on the same node $n$,
then the input types of these commands must match, i.e. $[x]=[z]$.
\end{mylem}
\begin{proof}
This result is similar to \cref{equiv_simple_same_commands} and
is easily shown using a proof by contradiction: if $[x]\neq [z]$, then there is no filesystem that
either $\cxynv$ or $\czwnv$ would not break, 
and consequently $S$ and $T$ cannot work on the same filesystem.
\end{proof}

