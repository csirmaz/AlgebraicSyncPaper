
\subsection{Filesystems}

We model filesystems using functions with a set of potential filesystem paths or \emph{nodes} ($\setn$) as their domain,
and a set of possible contents or values ($\setv$) as their codomain.
$\setn$  serves as a namespace for the filesystem:
it contains all possible nodes, including the ones 
where the file system contains no file or directory,
and so for most filesystems it is infinite.
Nodes in $\setn$ are also arranged in trees by a parent function ($\parent$) as described in \cref{def_noderel}.

\begin{mydef}[Filesystems, $\FS$]
A filesystem $\FS$ is a function
mapping the set of nodes $\setn$ to values in $\setv$:
\[ \FS: \setn \rightarrow \setv. \]
\end{mydef}

% Ordering nodes

The tree-like structure of $\setn$ is determined by
an ancestor / descendant relation defined over the nodes
which arranges them in a disjoint union of rooted directed trees,
and which can be derived from the partial function $\parent$ yielding
the parent of a node in $\setn$ provided it exists.
Tao et al. \cite{TSR} describe a similar filesystem model, although
they also model inodes and restrict the filesystem to just a single tree.

\begin{mydef}[Ordering on $\setn$: $\parent$, $\descendant$, $\descendantEq$, $\unrel$]\mlabel{def_noderel}
The partial function $\parent:\setn\nrightarrow\setn$
returns the parent node of $n$,
and is undefined if $n$ is the root of a tree.

The \emph{ancestor} / \emph{descendant} relation $\descendant$ is the
strict partial ordering determined by the $\parent$ function.
We write $n\descendant m$, or $n$ is the ancestor of $m$,
iff $n=\parent^i(m)$ for some integer $i\ge 1$.
% or the path name of $n$ is an initial segment of that of $m$. 
We write $n\descendantEq m$ iff $n\descendant m$ or $n=m$.

We write $n\unrel m$, or $n$ and $m$ are \emph{incomparable},
iff $n\not\descendantEq m$ and $n\not\ancestorEq m$;
that is, incomparable nodes are on different branches or on different trees.
% none of the path names describing their location is an initial segment of the other.
\end{mydef}
By assumption that the $\parent$ function does not induce loops, and so
$\descendant$ is indeed a strict partial ordering.

% Values

The combination of the set of nodes and the parent function,
$\langle \setn, \parent \rangle$, forms a \emph{skeleton}
which the filesystems populate with values from $\setv$.
As we require filesystem functions to be total,
we use a special value, $\empt\in\setv$, to indicate that the filesystem
is empty at a particular node, that is, there are no files or directories there.
We also assume that a filesystem has finitely many non-empty nodes,
and we consider any kind of metadata to be part of the values in $\setv$.

\begin{mydef}[Filesystem values: $\vald$, $\valfx$, $\empt$, $\eqclass{v}$]
The set of values $\setv$ is partitioned into directories, files, and the empty value $\empt$.
As usual, we write $[v]$ for the the equivalence class of $v\in\setv$ according to this partition,
which represents the type of the value.
Specific directory and file values are denoted by $\vald$ and $\valfx$, respectively,
and so the partitioning is in fact $[\vald], [\valfx], [\empt]$.
\end{mydef}

% Tree property

Every filesystem must have a so-called \emph{tree property}, which means that
if the filesystem is not empty at a node, and the node has a parent,
then there must be a directory at the parent node.
Using the notation introduced above, we can formally express this as follows.
\begin{mydef}[Tree property]\mlabel{def_treeprop}
A filesystem $\FS$ has the tree property iff
\[ \forall n\in\setn:
\FS(n) \neq \empt \: \Longrightarrow \: \Big[\FS\big(\parentf{n}\big)\Big] = [\vald] \]
wherever $\parent(n)$ is defined, that is, $n$ has a parent.
\end{mydef}

% SINGLEDIR

An essential additional assumption in our model is that there is only
a \emph{single directory value} ($|\setd|=1$).
The realization that
this creates a symmetry between directories and empty nodes
and makes it possible to exploit a hidden symmetry in filesystem commands
that allows formulating a correct and complete reconciliation algorithm
is one of the main contributions of this paper.
In \cref{sec_multidir} we explore the applicability of our model given this assumption;
specifically,
why applicability may not be affected and ways to relax this restriction.

We also assume that there are multiple different file contents, that is, $|[\valfx]|>1$.

% Notation

In the rest of paper
we fix the skeleton $\langle\setn,\parent\rangle$,
and the set of values $\setv$.
$\FS$ (with or without indices) denotes a filesystem,
and $n$, $m$ and $o$ are nodes in $\setn$.
