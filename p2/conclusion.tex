
\section{Conclusions and Further Research}\mlabel{sec_conclusion}

Apart from constructing a filesystem-free algebra,
there are many other ways in which further work can extend the current results.
An important extension would be to
consider reconciling not only two, but more replicas in a single step and
prove the correctness and maximality of the algorithm proposed,
or show that it is impossible to satisfy these criteria.

A related problem is to extend the system and the proofs
to allow for cases where reconciliation cannot
complete fully
or if only a subset of the replicas are reconciled 
(e.g. due to network partitioning),
both of which would result in a state where different replicas
have different common ancestors, that is,
the updates specific to the replicas start from different points
in the update history of the filesystems.
Existing research can offer pointers as to how such cases can be modeled
in our algebraic system.
Parker et al. \cite{PPRS} and Cox and Josephson \cite{CJ}
describe version vectors (update histories) kept as metadata,
while Chong and Hamadi present distributed algorithms that allow incremental synchronization \cite{CH}.
Representing individual updates to files
in their modification histories (as described in \cite{CJ})
as separate commands could also enable an algebraic synchronizer to reconcile otherwise
conflicting updates and resolve partial reconciliations.

Future work could also investigate extending the 
model and algorithms to the $|\setd|>1$ case
so that directory metadata could be represented directly
as opposed to through pre- and post-processing as described
in \cref{sec_multidir}.

And finally, we hope that this work, together with \cite{NREC}, provides
a blueprint of constructing an algebra of commands for different storage protocols
(e.g. XML trees, mailbox folders, generic relational databases, etc.),
and of demonstrating the adequacy and completeness of the update and conflict detection and reconciliation
algorithms defined over it.
This, in turn, can offer formal verification of the algorithms underlying
specific implementations in a variety of synchronizers.
Alternatively, by generalizing the parent--child relationships between filesystem nodes,
the demonstrated properties of minimal sequences of commands
and domains of sets of sequences ($\worksmeqsign$)
may also contribute to future research into algebraic structures
constrained by predefined sparse connections between their elements.
