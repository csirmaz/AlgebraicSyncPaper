
\section{Extending the Synchronizer for Directory Metadata}\mlabel{sec_multidir}

As suggested above, one of the central assumptions of our model is that
there is only one directory value, that is, directories are not differentiated
by the meta-information they contain.
In this section we describe reasons why this may not limit the applicability
of even the current model, and we describe pre- and post-processing steps
to overcome this limitation.

We note that commercial designs for synchronizers 
often avoid considering metadata in directories as
as these are generally not understood well by users,
and, if needed, conflict resolution on these settings can be easily automated.
See, for example, the seminal work of Balasubramaniam and Pierce and the Unison synchronizer \cite{BP} or \cite{BZ}.

Despite the above, there can be applications where directory metadata
is considered important.
Synchronizers based on our model can be readily extended to handle them
by duplicating $\setn$. 
Given the input filesystems, we add a special node as an extra child under each original node,
and we encode directory metadata in a file under each directory.
It is easy to see that update detection and conflict resolution can continue as expected,
with the only exception of a potential conflict detected on one of these special nodes.
In these cases the synchronizer may prescribe creating a directory without creating
the special metadata file, which is clearly not possible to do on the target filesystem,
as creating a directory entails specifying its metadata as well.
In these cases, the synchronizer can either fall back to default values and flag the issue
for review later, or if, as suggested above, conflicts on the metadata (e.g. readable, writeable and executable flags)
can be easily automated, then it could form part of the implementation.
