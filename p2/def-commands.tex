
\subsection{Commands on Filesystems}

% TODO identity command instead of the empty sequence?

Next we define commands on filesystems.
As described above, we aimed to select a set of commands
that captures as much information
about the operations as possible, and is also symmetric.
We start by summarizing our reasons for doing so.

Let us consider what kind of information is usually encoded in filesystem operations.
A minimal set of commands, based on the most frequent tools implemented by filesystems,
may be $create(n,\valv)$, which creates a file or directory ($\valv$) at the empty node $n$;
$edit(n,\valv)$, which replaces the earlier value at $n$ with $\valv$;
and $remove(n)$, which deletes the value at $n$---where 
$n\in\setn$ and $\valv\in\setv$ (but $\valv\neq\empt$).
Regarding their output, that is, the state of the filesystem at $n$
after applying the command,
we know that after $create$ or $edit$, $\FS(n)\neq\empt$, whereas after $remove$,
$\FS(n)$ will be $\empt$. 
However, from \cite{NREC} and \cite{CBNR} we know that a useful set of axioms
will in some cases need to distinguish between, for example,
$edit$s that result in directories ($edit(n,\vald)$) and
ones that result in files ($edit(n,\valf)$), and treat them as separate commands,
as their behaviors are quite different when combined with other commands.
Indeed, Bill Zissimopoulos' work
demonstrated \cite{BZ}
that extending this distinction to more commands ultimately simplifies
the definition of conflicting commands, as our model will then able to predict the behavior of commands
more precisely.
In other words, encoding the type of the output ($\cchard$, $\ccharf$ or $\ccharb$) in the commands is definitely useful.
At the same time, there is never any need to consider the
exact output \emph{value} of a command,
as the success or failure of filesystem commands only depends on the types of values in the filesystem.

Notice, however, that the commands listed above also encode some information about 
their input, the state of the filesystem
before the command is applied. In particular, $create(n,\valv)$ requires that there are no files
or directories at $n$, while $edit(n,\valv)$ and $remove(n)$ require the opposite.
This creates an arbitrary asymmetry where
there is now more information available about their output than about their input.
As, based on the above, we expect that encoding more information in the commands
results in a model with greater predictive powers,
and in order to resolve this asymmetry, 
we propose a set of commands that encode
the type of the original state of $\FS(n)$ as well.
(Some real-life filesystem commands like $rmdir$ do this already.)

\begin{mydef}[Regular commands]
Therefore we have four pieces of information related to each command,
and represent them using quadruples that list
their input type ($X$), output type ($Y$),
the node to which they are applied ($n$),
and the value they store at the node ($\valvy$):
\[ \cxynv, \]
where $\valvy\in\setvx{Y}$.

We use $\cchard$, $\ccharf$ or $\ccharb$ to represent the different
types, and omit the final value if it is determined by the output type
(when it is $\empt$ or $\vald$).
For example, $\cbdaa{n}{\vald}$ will represent $mkdir(n,\vald)$,
and $\cdba{n}$ will represent $rmdir(n)$.
\end{mydef}

\bigskip

\noindent
A command can only succeed if the original value at node $n$ has a type that matches
the input type of the command. If this is not the case, or if the resulting
filesystem no longer has the tree property, then we say that the command
{\em breaks} the filesystem, and assigns $\fsbroken$ to it.
Broken filesystems are considered equal, but not equal to any working filesystem.

So that we could reason about sequences of commands that break every filesystem, 
we introduce, in addition to the regular commands,
the command $\cbrk$ that simply breaks every filesystem.

We note that a command or a sequence of commands is applied to a filesystem
by prefixing the command or sequence to it, for example: $\cbrk\aFS$, $\cbdaa{n}{\vald}\aFS$, 
or simply $S\aFS$ if $S$ is a sequence of commands.

To define the exact effect of the commands we use
the \emph{replacement operator}  which changes the
value of the filesystem at a given point only.

\begin{mydef}[Effect of commands]
We use a \emph{replacement operator} of the form
$\FS[\valv/n]$ to denote a filesystem derived from $\FS$:
\[ \FS[\valv/n](m) =
   \begin{cases}
   \valv &\mbox{if~} m=n\\
   \FS(m) &\mbox{otherwise.}
   \end{cases}
\]
The effect of the commands is then as follows:
\begin{align*}
&\cbrk\aFS = \fsbroken \\
&\cxynv\aFS = 
   \begin{cases}
   \fsbroken &\mbox{if~} \FS=\fsbroken\\
   \fsbroken &\mbox{if~} \FS(n)\not\in\setvx{X}\\
   \fsbroken &\mbox{if~} \FS[\valvy/n] \mbox{~violates the tree property}\\
   \FS[\valvy/n] &\mbox{otherwise.}
   \end{cases}
\end{align*}
\end{mydef}

In general, when we describe multiple or unknown commands, we may substitute one or both
of their types with variables, as we did with $\cxynv$. In this context,
if we write that $\cxy=\czw$, we mean that the input and output types
of the commands are the same ($X=Z$ and $Y=W$), while a full equality 
($\cxynv=\czwmv$)
implies
that their nodes and output values are also the same.

% Simplification
% --------------

\bigskip

\noindent
For reasons also listed in \cite{NREC}, in this paper we will not consider
a $move$ or $rename$ command. Regarding the theoretical reasoning we aim to follow,
this turns out to be useful because this would be the only command that affects
filesystems at two nodes at once, therefore describing 
the dependencies for $move$ would call for a more complicated model.
From a pragmatic perspective, this restriction does not mean that in an application
implementing conflict resolution using the algorithm described here would not be
able to handle renames by pre- and post-processing changes in the filesystem to
discover them, which can enhance usability, but which
(especially when a rename is combined with changes to the content)
is a non-trivial problem in itself.
